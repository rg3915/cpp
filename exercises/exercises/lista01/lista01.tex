\documentclass[a4paper]{article}
\usepackage[utf8]{inputenc}
\usepackage{../../../../../LaTeX/modelos/preambulo}
\usepackage{hyperref}
\hypersetup{pdfpagelayout=SinglePage,%TwoPageLeft,
  bookmarksopen=true,
  colorlinks=true,
  urlcolor=blue,
  linkcolor=black,
  pdftitle={Introdução a Programação},
  pdfauthor={R\'egis S. Santos}
}

%% preambulo para intro_prog.tex
\lstset{explpreset={
    language=C++,
    float=hbp,
    basicstyle=\ttfamily\small, 
    identifierstyle=\color{colIdentifier}, 
    keywordstyle=\color{colKeys}, 
    stringstyle=\color{colString}, 
    commentstyle=\color{colComments}, 
    columns=flexible, 
    tabsize=4, 
    frame=none,   %%frame=single
    rframe={}
    },
    language=C++,
    basicstyle=\ttfamily\small, 
    identifierstyle=\color{colIdentifier}, 
    keywordstyle=\color{colKeys}, 
    stringstyle=\color{colString}, 
    commentstyle=\color{colComments}, 
    hsep=10mm,
    extendedchars=true, 
    showspaces=false, 
    showstringspaces=false, 
    %%numbers=none,
    numbers=left,
    numberstyle=\tiny, 
    breaklines=true, 
    backgroundcolor=\color{green!10}, 
    breakautoindent=true, 
    captionpos=b,
    xleftmargin=0pt,
}

\fancyfoot[C]{\scriptsize{Lista 01}}

\theorembodyfont{\normalfont\upshape}
\newtheorem{eex}{Exerc\'icio E-}

\title{Introdução à Programação - turma 1 (diurno)}
\author{Lista 01 - R\'egis S. Santos}
\date{2012}
%*******************************************************
\begin{document}

\maketitle

%1
\begin{eex}
* Dada uma sequ\^encia de n\'umeros inteiros positivos seguida por 0, imprimir seus quadrados.
\end{eex}

\begin{lstlisting}
#include <iostream>
using namespace std;
int main()
{
    int n, quadrado = 0;
    cout << "Digite um numero: ";
    cin >> n;
    quadrado = n * n;
    cout << quadrado << endl;
    while (n != 0)
    {
        cout << "Digite o proximo numero: ";
        cin >> n;
        quadrado = n * n;
        cout << quadrado << endl;
    }
    return 0;
}
\end{lstlisting}

%2
\begin{eex}
Dados o n\'umero $n > 0$ de alunos de uma turma e suas notas da primeira prova, determinar a maior nota, a menor nota e a m\'edia inteira dessa prova.
\end{eex}

\begin{lstlisting}
#include <iostream>
using namespace std;
int main()
{
    int n, i, nota, maior, menor, media;
    cout << "Digite o numero de alunos: ";
    cin >> n;
    cout << "Digite as notas: ";
    cin >> nota;
    maior = nota;
    menor = nota;
    media = nota;
    for (i = 1; i < n; i++)
    {
        cin >> nota;
        if (nota > maior)
            maior = nota;
        if (nota < menor)
            menor = nota;
        media += nota;
    }
    media = media / n;
    cout << "A maior nota 'e " << maior << endl;
    cout << "A menor nota 'e " << menor << endl;
    cout << "A media 'e " << media << endl;
    return 0;
} 
\end{lstlisting}

%3
\begin{eex}
Dado um n\'umero natural na base bin\'aria, transform\'a-lo para a base decimal. Exemplo: Dado 110010 a saída será 50, pois $1.2^5 + 1.2^4 + 0.2^3 + 0.2^2 + 1.2^1 + 0.2^0 = 50$.
\end{eex}

\begin{lstlisting}
#include <iostream>
using namespace std;
int main()
{
    int binario, potencia = 1, resto = 0, decimal = 0;
    cout << "Digite um numero binario: ";
    cin >> binario;
    while (binario > 0)
    {
        resto = binario % 10;
        binario /= 10;
        decimal = decimal + (resto * potencia);
        potencia *= 2;
    }
    cout << decimal << endl;
    return 0;
}
\end{lstlisting}

%4
\begin{eex}
* Dado um n\'umero natural na base decimal, transormá-lo para a base binária. Exemplo: Dado 50 a saída deverá ser 110010.
\end{eex}

\begin{lstlisting}
#include <iostream>
using namespace std;
int main()
{
    int n, binario = 0, digito, potencia = 1;
    cout << "Digite um numero: ";
    cin >> n;
    while (n > 0)
    {
        digito = n % 2;
        n = n / 2;
        binario = binario + digito * potencia;
        potencia = potencia * 10;
    }
    cout << binario << endl;
    return 0;
} 
\end{lstlisting}

%5
\begin{eex}
Dados um inteiro positivo $n$ e $n$ sequ\^encias de números inteiros, cada qual terminada por 0, calcular a soma dos números pares de cada sequ\^encia.
\end{eex}

\begin{lstlisting}
#include <iostream>
using namespace std;
int main ()
{
    int n, i, numero, soma = 0;
    cout << "Digite o comprimento da sequencia: ";
    cin >> n;
    for (i = 0; i < n; i++)
    {
        while (numero != 0)
        {
            cin >> numero;
            if (numero % 2 == 0)
            {
                soma += numero;
            }
        }
        cout << "Soma = " << soma << endl;
        numero = 1; /*acho que fiz uma gambiarra!*/
        soma = 0;
    }
    return 0;
}
\end{lstlisting}

%6
\begin{eex}
* Dados $n$ e uma sequ\^encia de $n$ números inteiros positivos, calcular a soma dos números da sequ\^encia que s\~ao primos.
\end{eex}

\begin{lstlisting}
#include <iostream>
using namespace std;
int main()
{
    int n, i, j, numero = 0, primo = 0, soma = 0;
    cout << "Digite o comprimento da sequencia: ";
    cin >> n;
    cout << "Digite os " << n << " numeros: ";
    for (i = 0; i < n; i++)
    {
        cin >> numero;
        for (primo = 0, j = 1; j <= numero; j++)
        {
            if (numero %j == 0)
                primo ++;
        }
        if (primo == 2)
        {
            soma = soma + numero;
        }
    }
    cout << "A soma dos primos 'e = " << soma << endl;
    return 0;
}
\end{lstlisting}

\end{document}