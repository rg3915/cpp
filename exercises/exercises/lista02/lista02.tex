\documentclass[a4paper]{article}
\usepackage[utf8]{inputenc}
\usepackage{../../../../../LaTeX/modelos/preambulo}
\usepackage{../../../prelista}
\usepackage{hyperref}
\hypersetup{pdfpagelayout=SinglePage,%TwoPageLeft,
  bookmarksopen=true,
  colorlinks=true,
  urlcolor=blue,
  linkcolor=black,
  pdftitle={Introdução a Programação},
  pdfauthor={R\'egis S. Santos}
}

\fancyfoot[C]{\scriptsize{Lista 02}}

\theorembodyfont{\normalfont\upshape}
\newtheorem{eex}{Exerc\'icio E-}

\title{Introdução à Programação - turma 1 (diurno)}
\author{Lista 02 - R\'egis S. Santos}
\date{2012}
%*******************************************************
\begin{document}

\maketitle

%8
\setcounter{eex}{7}
\begin{eex}
* Escolha 10 problemas vistos em sala de aula e faça um programa em C++ para cada um deles. Você pode copiar o programa feito no quadro.
\end{eex}

\begin{sol}
A seguir os exercícios solicitados com enunciado e resposta, observe o nome de cada arquivo logo no início de cada enunciado.

\

(1) \textbf{E08prob01.cpp}

Dada uma sequência de números inteiros diferentes de zero, terminada por um zero, calcular a sua soma. Por exemplo, para a sequência 12 7 4 -6 8 0 o seu programa deve escrever o número 35.

\begin{lstlisting}
#include <iostream>
using namespace std;
int main()
{
    int numero, soma;
    soma = 0;
    cin >> numero;
    cout << numero << endl;
    while (numero != 0)
    {
        soma = soma + numero;
        cin >> numero;
        cout << numero << endl;
    }
    cout << "soma = " << soma << endl;
    return 0;
}
\end{lstlisting}

(2) \textbf{E08prob02.cpp}

Dados os números inteiros $n$ e $k$ com $k \mai 0$, determinar $n^k$. Por exemplo, dados os números $3$ e $4$ o programa deve escrever o número $81$.

\begin{lstlisting}
#include <iostream>
using namespace std;
int main()
{
    int n, k, resultado, potencia;
    cout << "Digite um numero:";
    cin >> n;
    cout << "Digite outro numero:";
    cin >> k;
    resultado = 1;
    potencia = 0;
    while (potencia < k)
    {
        resultado = resultado * n;
        potencia = potencia + 1;
    }
    cout << resultado << endl;
    return 0;
}
\end{lstlisting}

(3) \textbf{E08prob03.cpp}

Dado um número inteiro $n \mai 0$, calcular $n!$.

\begin{equation*}
  n! =
  \begin{cases}
    1 		& \mbox{, se } n = 0 \\
    n 		& \mbox{, se } n = 1 \\
    n(n - 1)! 	& \mbox{, se } n > 1
  \end{cases}
\end{equation*}

\begin{lstlisting}
#include <iostream>
using namespace std;
int main()
{
    int n, resultado, i;
    cout << "Digite um numero:";
    cin >> n;
    resultado = 1;
    i = n;
    while (i >= 1)
    {
        resultado = resultado*i;
        i = i - 1;
    }
    cout << resultado << endl;
    return 0;
}
\end{lstlisting}

(4) \textbf{E08prob05.cpp}

Dados um número inteiro $n \mai 0$, e uma sequência com $n$ inteiros, determinar a soma dos inteiros positivos e a soma dos inteiros negativos da sequência. Por exemplo, para a sequência 6 -2 7 0 -5 8 4 o programa deve escrever o número 19 e -7.

\begin{lstlisting}
#include <iostream>;
using namespace std;
int main()
{
    int n, i, numero, pos, neg;
    i = 0;
    pos = 0;
    neg = 0;
    cout << "Digite o comprimento da sequencia: ";
    cin >> n;
    while (i < n)
    {
        cout << "Digite o proximo numero: ";
        cin >> numero;
        if (numero > 0)
        {
            pos = pos + numero;
        }
        else
        {
            neg = neg + numero;
        }
        i = i + 1;
    }
    cout << "A soma dos inteiros positivos eh: " << pos << endl;
    cout << "A soma dos inteiros negativos eh: " << neg << endl;
    return 0;
}
\end{lstlisting}

(5) \textbf{E08prob06.cpp}

Dados um número inteiro $n \mai 0$, e uma sequência com $n$ inteiros, determinar quantos números da sequência são pares e quantos são ímpares. Por exemplo, para a sequência 6 -2 7 0 -5 8 4 o programa deve escrever o número 4 para o número de pares 2 para o número de ímpares.

\begin{lstlisting}
#include <iostream>
using namespace std;
int main()
{
    int n, i, numero, par, impar;
    i = 0;
    par = 0;
    impar = 0;
    cout << "Digite o comprimento da sequencia: ";
    cin >> n;
    while (i < n)
    {
        cout << "Digite o proximo numero: ";
        cin >> numero;
        if (numero % 2 == 0)
        {
            par = par + 1;
        }
        else
        {
            impar = impar + 1;
        }
        i = i + 1;
    }
    cout << "A sequencia possui " << par << " inteiros pares e " << impar << " inteiros impares." << endl;
    return 0;
}
\end{lstlisting}

(6) \textbf{E08probextra01.cpp}

Dados um número inteiro $n \mai 0$, e um digito $d$ $(0 <= d <= 9)$ determinar quantas vezes $d$ ocorre em $n$. Por exemplo, para $n = 63453$ e $d = 3$ o programa deve imprimir $2$.

\begin{lstlisting}
#include <iostream>
using namespace std;
int main()
{
    int n, d, resto;
    resto = 0;
    cout << "Digite um numero: ";
    cin >> n;
    cout << "Digite um digito: ";
    cin >> d;
    while (n > 0)
    {
        if (n % 10 == d)
        {
            resto = resto + 1;
        }
        n = n / 10;
    }
    cout << "O numero " << d << " aparece "<< resto << " vezes."<< endl;
    return 0;
}
\end{lstlisting}

(7) \textbf{E08prob08.cpp}

Dados um número inteiro $n > 0$ e as notas de $n$ alunos, determinar quantos alunos ficaram de recuperação. Um aluno está de recuperação se sua nota estiver entre $30$ e $50$ (nota máxima $100$).

\begin{lstlisting}
#include <iostream>
using namespace std;
int main(){
    int n, i, nota, rec;
    cout << "Digite o numero de alunos: ";
    cin >> n;
    rec = 0;
    for (i = 0; i < n; i = i + 1)
    {
        cout << "Digite as notas: ";
        cin >> nota;
        if (nota >= 30)
        {
            if (nota <= 50)
            {
                rec = rec + 1;
            }
        }
    }
    cout << rec << " alunos estao de recuperacao." << endl;
    return 0;
}
\end{lstlisting}

(8) \textbf{E08prob11.cpp}

Dado um número inteiro $n > 0$, verificar se $n$ é primo.

\begin{lstlisting}
#include <iostream>
using namespace std;
int main()
{
    int n, i, primo = 1;
    cout << "Digite um numero: ";
    cin >> n;
    if (n == 1)
    {
        primo = 0;
    }
    for (i = 2; i < n; i++)
    {
        if (n % i == 0)
        {
            primo = 0;
        }
    }
    if (primo == 1)
    {
        cout << n << " 'e primo." << endl;
    }
    else
    {
        cout << n << " nao 'e primo." << endl;
    }
    return 0;
}
\end{lstlisting}

(9) \textbf{E08prob14.cpp}

Dados um inteiro $n > 0$, e uma sequência com $n$ inteiros, verificar se a sequência é uma progressão aritmética.

\begin{lstlisting}
#include <iostream>
#define TRUE 1
#define FALSE 0

using namespace std;
int main()
{
    int n, i, anterior, atual, razao = 0, res = TRUE;
    cout << "Digite o comprimento da sequencia: ";
    cin >> n;
    cin >> anterior;
    if (n > 1)
    {
        cin >> atual;
        razao = atual - anterior;
    }
    for (i = 2; i < n; i++)
    {
        anterior = atual;
        cin >> atual;
        if (atual - anterior != razao)
            res = FALSE;
    }
    if (res) /*ou res = TRUE*/
        cout << "'E uma P.A." << endl;
    else
        cout << "Nao 'e uma P.A." << endl;
    return 0;
}
\end{lstlisting}

(10) \textbf{E08prob15.cpp}

Sabe-se que cada número da forma $n^3$ é igual a soma de $n$ ímpares consecutivos. Exemplos:

\[
\begin{gathered}
  1^3 = 1 \hfill \\
  2^3 = 3 + 5 \hfill \\ 
  3^3 = 7 + 9 + 11 \hfill \\ 
  4^3 = 13 + 15 + 17 + 19 \hfill \\ 
\end{gathered} 
\]

Dado um inteiro $m > 0$, determinar os ímpares consecutivos cuja soma é igual a $n^3$, para $n$ assumindo valores de $1$ a $n$.

\begin{lstlisting}
#include <iostream>
using namespace std;
int main()
{
    int m, n, i, candidato, termo;
    cout << "Digite um numero: ";
    cin >> m;
    for (n = 1; n <= m; n++)
    {
        candidato = n*n - n + 1;
        cout << n << "^3 = ";
        termo = candidato;
        for (i = 0; i < n - 1; i++)
        {
            cout << termo << " + ";
            termo += 2;
        }
        cout << termo << endl;
    }
    return 0;
}
\end{lstlisting}

\end{sol}

\newpage 

%9
\begin{eex}
Dizemos que um número natural $n$ com pelo menos dois algarismos é \emph{palíndrome} se o primeiro algarismo de $n$ é igual ao seu último algarismo, o segundo algarismo de $n$ é igual ao penúltimo algarismo, e assim por diante. Exemplos: 567765 e 32423 são palíndromes e 567675 não é palíndrome. 

Dado um inteiro $n, n \mai 10$, verificar se $n$ é palíndrome.
\end{eex}

\begin{sol}
\begin{lstlisting}
#include<iostream>
using namespace std;
int main()
{
    int n, resto, soma = 0, temp;
    cout << "Digite um numero: ";
    cin >> n;
    temp  = n;
    while(n != 0)
    {
         resto = n % 10;
         n = n / 10;
         soma = soma * 10 + resto;
    }
    if(temp == soma)
         cout << "'e palindrome" << endl;
    else
         cout << "nao 'e palindrome" << endl;
    return 0;
}
\end{lstlisting}
\end{sol}

%10
\begin{eex}
Dado um inteiro positivo $n$, determine o valor dado pela seguinte equação:

\myfuncS{F_n}
{\sum_{k=1}^n \frac{1}{k^2}}
{n \text{ é ímpar}}
{\sum_{k=1}^{\mfrac{n}{2}}\frac{1}{2^k}}
{n \text{ é par}}
\end{eex}

\begin{sol}
\begin{lstlisting}
#include <iostream>
using namespace std;
int main()
{
    int n, i, j, pot = 1;
    float soma = 0.0;
    cout << "Digite um numero: ";
    cin >> n;
    /* verifica se n 'e par */
    if (n % 2 == 0)
    {
        /* \sum 1/k^2 */
        for (i = 1; i <= n; i++)
        {
            pot = i * i;
            soma += 1.0 / pot;
        }
    }
    else
    {
        /* \sum 1/2^k */
        for (i = 1; i <= n / 2; i++)
        {
            pot = 1;
            for (j = 1; j <= i; j++)
                pot *= 2;
            soma += 1.0 / pot;
        }
    }
    cout << soma << endl;
    return 0;
}
\end{lstlisting}
\end{sol}

%11
\begin{eex}
Dizemos que um número $i$ é congruente a $j$ módulo $m$ se $i\% m = j \% m$. Exemplo: $35$ é congruente a $39$ módulo $4$, pois $35\% 4 = 3 = 39 \% 4$.

Dados inteiros positivos $n, j$, e $m$, imprimir os $n$ primeiros naturais congruentes a $j$ módulo $m$.
\end{eex}

\begin{sol}
\begin{lstlisting}
#include <iostream>
using namespace std;
int main()
{
    int m, n, i, j;
    cout << "Digite tres numeros: ";
    cin >> n >> j >> m;
    for (i = 0; i <= n; i++)
        if (i % m == j % m)
            cout << i << " ";
    cout << endl;
    return 0;
}
\end{lstlisting}
\end{sol}

\newpage 

%12
\begin{eex}
Dado um inteiro $n$, calcular o valor da seguinte soma:

\[
\frac{1}{n} + \frac{2}{n-1} + \frac{3}{n-2} + \ldots + \frac{n}{1}
\]
\end{eex}

\begin{sol}
\begin{lstlisting}
#include <iostream>
using namespace std;
int main()
{
    int n, i, k;
    float sn = 0.0;
    cout << "Digite um numero: ";
    cin >> n;
    k = n;
    for (i = 1; i <= n; i++)
    {
        sn = sn + (float)i / k;
        k--;
    }
    cout << sn << endl;
    return 0;
}
\end{lstlisting}
\end{sol}

%13
\begin{eex}
* Faça um programa que calcula a soma

\[
1 - \frac{1}{2} + \frac{1}{3} - \frac{1}{4} + \ldots + \frac{1}{9999} - \frac{1}{10000}
\]

pelas seguintes maneiras:

\begin{enumerate}[a)]
 \item adição dos termos da direita para a esquerda;
 \item adição dos termos da esquerda para a direita;
 \item adição separada dos termos positivos e dos termos negativos da esquerda para a direita;
 \item adição separada dos termos positivos e dos termos negativos da direita para a esquerda.
\end{enumerate}

Compare e discuta os resultados obtidos no computador.
\end{eex}

\begin{sol}
\textbf{E13p1.cpp}

\begin{lstlisting}
#include <iostream>
using namespace std;
int main()
{
    /* d = denominador */
    int i, d;
    float soma = 0.0;
    for (i = 10000; i > 0; i--)
    {
        /* verifica se i 'e par. */
        if (i % 2 == 0)
            d = -i;
        else
            d = i;
        soma += 1.0 / d;
    }
    cout << soma << endl;
    return 0;
}
\end{lstlisting}

\textbf{E13p2.cpp}

\begin{lstlisting}
#include <iostream>
using namespace std;
int main()
{
    /* d = denominador */
    int i, d;
    float soma = 0.0;
    for (i = 1; i <= 10000; i++)
    {
        /* verifica se i 'e par. */
        if (i % 2 == 0)
            d = -i;
        else
            d = i;
        soma += 1.0 / d;
    }
    cout << soma << endl;
    return 0;
}
\end{lstlisting}

\textbf{E13p3.cpp}

\begin{lstlisting}
#include <iostream>
using namespace std;
int main()
{
    int i;
    float somaPos = 0.0, somaNeg = 0.0;
    for (i = 1; i <= 10000; i++)
    {
        /* verifica se i 'e par. */
        if (i % 2 == 0)
            somaNeg += 1.0 / -i;
        else
            somaPos += 1.0 / i;
    }
    cout << "soma positivos: " << somaPos << endl;
    cout << "soma negativos: " << somaNeg << endl;
    cout << "soma total: " << somaPos + somaNeg << endl;
    return 0;
}
\end{lstlisting}

\textbf{E13p4.cpp}

\begin{lstlisting}
#include <iostream>
using namespace std;
int main()
{
    int i;
    float somaPos = 0.0, somaNeg = 0.0;
    for (i = 10000; i > 0; i--)
    {
        /* verifica se i 'e par. */
        if (i % 2 == 0)
            somaNeg += 1.0 / -i;
        else
            somaPos += 1.0 / i;
    }
    cout << "soma positivos: " << somaPos << endl;
    cout << "soma negativos: " << somaNeg << endl;
    cout << "soma total: " << somaPos + somaNeg << endl;
    return 0;
}
\end{lstlisting}

Observe os resultados obtidos:

\textbf{E13p1}: 0.693097

\textbf{E13p2}: 0.693091

\

\textbf{E13p3}:

soma positivos: 5.24036

soma negativos: -4.54726

soma total: 0.693102

\

\textbf{E13p4}:

soma positivos: 5.24035

soma negativos: -4.54725

soma total: 0.693098

Os resultados sofrem alguma alteração depois de algumas casas decimais.

\end{sol}



\end{document}