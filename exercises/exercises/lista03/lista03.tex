\documentclass[a4paper]{article}
\usepackage[utf8]{inputenc}
\usepackage{../../../../../LaTeX/modelos/preambulo}
\usepackage{../../../prelista}
\usepackage{hyperref}
\hypersetup{pdfpagelayout=SinglePage,%TwoPageLeft,
  bookmarksopen=true,
  colorlinks=true,
  urlcolor=blue,
  linkcolor=black,
  pdftitle={Introdução a Programação},
  pdfauthor={R\'egis S. Santos}
}

\fancyfoot[C]{\scriptsize{Lista 03}}

\theorembodyfont{\normalfont\upshape}
\newtheorem{eex}{Exerc\'icio E-}

\title{Introdução à Programação - turma 1 (diurno)}
\author{Lista 03 - R\'egis S. Santos}
\date{2012}
%*******************************************************
\begin{document}

\maketitle

%14
\setcounter{eex}{13}
\begin{eex}
Dizemos que um inteiro positivo $n$ é \emph{perfeito} se for igual à soma de seus divisores positivos diferentes de $n$. Exemplo: 6 é perfeito, pois $1 + 2 + 3 = 6$. Faça um programa em C++ que verifica se um dado número inteiro positivo é perfeito.
\end{eex}

\begin{sol}
\begin{lstlisting}
#include <iostream>
using namespace std;
int main()
{
    int n, i, perfeito = 0;
    cout << "Digite um numero: " << endl;
    cin >> n;
    for (i = 1; i < n; i++)
    {
        if (n % i == 0)
            perfeito += i;
    }
    if (perfeito == n)
        cout << "'E perfeito." << endl;
    else
        cout << "Nao 'e perfeito." << endl;
    return 0;
}
\end{lstlisting}
\end{sol}

\newpage 

%15
\begin{eex}
Um matemático italiano da idade média conseguiu modelar o ritmo do crescimento da população de coelhos através de uma sequência de números naturais que passou a ser conhecida como \emph{sequência de Fibonacci}. O $n$-ésimo número da sequência de Fibonacci $F_n$ é dado pela seguinte fórmula de recorrência:

\[
\left\{ \begin{gathered}
  F_1  = 1 \hfill \\
  F_2  = 2 \hfill \\
  F_i  = F_{i - 1}  + F_{i - 2} , \quad \text{para } i \mai 3. \hfill \\ 
\end{gathered}  \right.
\]

Faça um programa em C++ que dado $n$ natural calcula $F_n$.
\end{eex}

\begin{sol}
\begin{lstlisting}
#include <iostream>
using namespace std;
int main()
{
    int n, i, anterior, atual, temp;
    anterior = atual = 1;
    cout << "Digite o n-esimo termo da sequencia de Fibonacci: ";
    cin >> n;
    for (i = 3; i <= n+1; i++)
    {
        temp = atual;
        atual += anterior;
        anterior = temp;
    }
    cout << atual << endl;
    return 0;
}
\end{lstlisting}
\end{sol}

\newpage 

%16
\begin{eex}
* Considere o conjunto $H = H_1 \cup H_2$ de pontos reais, onde

\[
H_1 = \{ (x,y) | x \mei 0, y \mei 0, y + x^2 + 2x - 3 \mei 0 \} \text{ e}
\]

\[
H_2 = \{ (x,y) | x \mai 0, y + x^2 - 2x - 3 \mei 0 \}.
\]

Faça um programa que lê um inteiro positivo $n$ e uma sequência de $n$ pontos reais $(x,y)$ e verifica se cada ponto pertence ou não ao conjunto $H$. O programa deve também contar o número de pontos da sequência que pertencem a $H$.
\end{eex}

\begin{sol}
\begin{lstlisting}
#include <iostream>
using namespace std;
int main()
{
    int n, i, pertence = 0;
    float x, y;
    cout << "Digite um numero: ";
    cin >> n;
    for (i = 0; i < n; i++)
    {
        cin >> x >> y;
        if ((x <= 0 && y <= 0 && y + x*x + 2*x - 3 <= 0) ||
            (x >= 0 && y + x*x - 2*x - 3 <= 0))
        {
            cout << "Pertence a H." << endl;
            pertence ++;
        }
        else
            cout << "Nao pertence a H." << endl;
    }
    cout << pertence << " pontos pertencem a H." <<endl;
    return 0;
}
\end{lstlisting}
\end{sol}

\newpage 

%17
\begin{eex}
* Dada uma frase terminada por ’.’, faça um programa em C++ que determina quantas letras e quantas palavras aparecem
no texto. Por exemplo, no texto “O voo GOL547 saiu com 10 passageiros.” há 25 letras e 7 palavras.
\end{eex}

\begin{sol}
\begin{lstlisting}
#include <iostream>
using namespace std;
int main()
{
    int palavras = 1, letras = 0;
    char c;
    cin >> noskipws >> c;
    while (c != '.')
    {
        if (c == ' ')
            palavras ++;
        if ((c >= 'a' && c <= 'z') || (c >= 'A' && c <= 'Z'))
            letras ++;
        cin >> noskipws >> c;
    }
    cout << palavras << " palavras." << endl;
    cout << letras << " letras." << endl;
    return 0;
}
\end{lstlisting}
\end{sol}

%18
\begin{eex}
Dada uma frase terminada por ’.’, faça um programa em C++ que determina quantos caracteres são divisíveis por 2 ou por 3.
Lembre que cada caracter possui um número decimal a ele associado.
\end{eex}

\begin{sol}
\begin{lstlisting}
#include <iostream>
using namespace std;
int main()
{
    int caracter = 0;
    char c;
    while (c != '.')
    {
        cin >> noskipws >> c;
        if (c % 2 == 0 || c % 3 == 0)
            caracter ++;
    }
    cout << caracter << endl;
    return 0;
}
\end{lstlisting}
\end{sol}

%19
\begin{eex}
Dado um número real $0 < \varepsilon < 1$, faça um programa em C++ que calcula uma aproximação do número de Euler (uma variante deste nome é número de Néper) através da série infinita:

\[
e = \SomaI[n]{0}{\frac{1}{n!}} = \frac{1}{{0!}} + \frac{1}{{1!}} + \frac{1}{{2!}} + \frac{1}{{3!}} +  \ldots  + \frac{1}{{k!}} +  \ldots 
\]

Inclua na aproximação todos os termos cujo valor absoluto é maior ou igual ao valor de $\varepsilon$.
\end{eex}

\begin{sol}
\begin{lstlisting}
#include <iostream>
using namespace std;
int main()
{
    int n, i, j;
    float fat, soma = 1.0;
    cout << "Digite um numero: " << endl;
    cin >> n;
    for (i = 1; i <= n; i++)
    {
        fat = 1;
        for (j = 1; j <= i; j++)
        {
            fat *= j;
        }
        soma += 1.0 / fat;
    }
    cout << soma << endl;
    return 0;
}
\end{lstlisting}
\end{sol}

\end{document}